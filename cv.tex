\documentclass[10pt,letter,sans]{moderncv}
\moderncvtheme[blue]{casual}
\usepackage[utf8]{inputenc}
\usepackage[scale=0.8]{geometry}
\recomputelengths

\firstname{Paolo Giovanni}
\familyname{Mezzano Barahona}
\address{Universidad de Santiago 1887}{Maip\'{u}, Santiago, RM}
\mobile{+56997875968}
\phone{+56232318152}
\email{pmezzano@gmail.com}
\social[github]{Cavallinux}
\nopagenumbers{}


\begin{document}
\maketitle

\section{Formaci\'{o}n acad\'{e}mica}
\cventry{2003 - 2007}{Ingenier\'{i}a de ejecuci\'{o}n en computaci\'{o}n e inform\'{a}tica}{Universidad de Santiago de Chile}{}{\textit{Santiago, RM}}{} %arguments 3 to 6 are optional

\section{Experiencia laboral}
\cventry{agosto 2016 - actualidad}{KDU}{Santiago}{Regi\'{o}n metropolitana}{Chile}{Actualmente, se trabaja en Centro de Negocios del Banco de Chile, en el desarrollo de su plataforma para
atencion de clientes, cuyos usuarios será\'{a}n ejecutivos de cuenta en sucursales y atención telef\'{o}nica remota} 
\cventry{septiembre 2012 - agosto 2016}{Experti}{Santiago}{Regi\'{o}n metropolitana}{Chile}{Se trabaj\'{o} en el desarrollo y mantenimiento de aplicaciones web para comercio electr\'{o}nico, usando JEE}
\cventry{agosto 2011 - agosto 2012}{Wiprojects Chile}{Santiago}{Regi\'{o}n metropolitana}{Chile}{Se trabaj\'{o} en el desarrollo de juegos multijugador, utilizando el framework de juegos Smartfox y utilizando los lenguajes ActionScript y Java para la generaci\'{o}n de las extensiones necesarias para cada juego.} % arguments 3 to 6 are optional
\cventry{septiembre 2010 - julio 2011}{Indra Sistemas Chile}{Santiago}{Regi\'{o}n metropolitana}{Chile}{Se trabaj\'{o} en el desarrollo de aplicaciones para telecomunicaciones, utilizando tanto Java, como C/C++ y Python}                % arguments 3 to 6 are optional
\cventry{diciembre 2009 - marzo 2010}{Imagemaker IT}{Santiago}{Regi\'{o}n metropolitana}{Chile}{Se trabaj\'{o} en el desarrollo de nuevas funcionalidades sobre la plataforma Weblogic 8.1, para el banco BCI, utilizando patrones de diseño JEE.}                % arguments 3 to 6 are optional
\cventry{agosto 2009 - noviembre 2009}{SIIGSA}{Santiago}{Regi\'{o}n metropolitana}{Chile}{Se trabaj\'{o} en el mantenimiento de un sitio web, hecho con JEE (framework Struts), encargado de llevar un cat\'{a}logo en línea de metadatos de la informaci\'{o}n territorial de las instituciones del Estado de Chile. Este cat\'{a}logo lo administra el Ministerio de Bienes Nacionales.}                % arguments 3 to 6 are optional
\cventry{junio 2009 - julio 2009}{IBM Chile}{Santiago}{Regi\'{o}n metropolitana}{Chile}{Se realiz\'{o} testing a aplicaci\'{o}n web, desarrollada con JEE, encargada del ciclo de vida de mantenimiento de la flota de aviones de LAN Chile.}                % arguments 3 to 6 are optional
\cventry{julio 2007 - enero 2009}{Rhiscom}{Santiago}{Regi\'{o}n metropolitana}{Chile}{Se realiz\'{o} entre otras labores, testing de aplicaciones de punto de venta IBM, y levantamiemto de requerimientos para aplicaciones de punto de venta IBM, siguiendo la metodología del Proceso Unificado. Junto con ello, se implement\'{o} una plataforma de seguimiento de bugs para los proyectos de Rhiscom, llamada Mantis.}                % arguments 3 to 6 are optional

\section{Idiomas}
\cvlanguage{Español}{Idioma nativo}{}

\newpage{}

\section{Competencias t\'{e}cnicas}
\cvcomputer{Sistemas operativos}{GNU/Linux, MS Windows}{Lenguajes de programaci\'{o}n}{ Java, C/C++, Python, ActionScript,Bash}
\cvcomputer{Bases de datos}{MySQL, PostgreSQL, Oracle, SQLServer}{Diseño web}{ X/HTML, XML, XSLT, XPath, CSS, JavaScript}
\cvcomputer{J2EE}{JSP/Servlets/JSTL, JSF, Struts, Spring, JMS, EJB 2.1/3.0, Hibernate}{Lenguajes de modelamiento}{UML}
\cvcomputer{Herramientas CASE}{Enterprise Architect,Power Designer}{IDEs}{Eclipse, Flash Develop, Netbeans, Visual Studio}
\cvcomputer{Construccion de software}{Maven, Ant}{Testing automatizado}{JUnit}
\cvcomputer{Integraci\'{o}n continua}{Bamboo, Jenkinks, JIRA}{}{}

\section{Otros intereses}
\cvline{}{Gusto sobremanera del f\'{u}tbol, el rock en todas sus variantes, el software libre y los entornos de escritorio libres (KDE, GNOME, LXDE)}
\end{document}
